%%%%%%%%%%%%%%%%%%%%%%%%%%%%%%%%%%%%%%%%%
% Short Sectioned Assignment
% LaTeX Template
% Version 1.0 (5/5/12)
%
% This template has been downloaded from:
% http://www.LaTeXTemplates.com
%
% Original author:
% Frits Wenneker (http://www.howtotex.com)
%
% License:
% CC BY-NC-SA 3.0 (http://creativecommons.org/licenses/by-nc-sa/3.0/)
%
%%%%%%%%%%%%%%%%%%%%%%%%%%%%%%%%%%%%%%%%%

%----------------------------------------------------------------------------------------
%	PACKAGES AND OTHER DOCUMENT CONFIGURATIONS
%----------------------------------------------------------------------------------------

\documentclass[paper=a4, fontsize=11pt]{scrartcl} % A4 paper and 11pt font size

\usepackage[T1]{fontenc} % Use 8-bit encoding that has 256 glyphs
%\usepackage{fourier} % Use the Adobe Utopia font for the document - comment this line to return to the LaTeX default
\usepackage[english]{babel} % English language/hyphenation
\usepackage[utf8]{inputenc}  %allows non-English characters
\usepackage{amsmath,amsfonts,amsthm} % Math packages
\usepackage{float}

\usepackage{sectsty} % Allows customizing section commands
%\allsectionsfont{\centering \normalfont\scshape} % Make all sections centered, the default font and small caps
\allsectionsfont{\centering}

\usepackage{fancyhdr} % Custom headers and footers
\pagestyle{fancyplain} % Makes all pages in the document conform to the custom headers and footers
\fancyhead{} % No page header - if you want one, create it in the same way as the footers below
\fancyfoot[L]{} % Empty left footer
\fancyfoot[C]{} % Empty center footer
\fancyfoot[R]{\thepage} % Page numbering for right footer
\renewcommand{\headrulewidth}{0pt} % Remove header underlines
\renewcommand{\footrulewidth}{0pt} % Remove footer underlines
\setlength{\headheight}{13.6pt} % Customize the height of the header

%\usepackage{geometry}
%\usepackage{pdflscape}


%\numberwithin{equation}{section} % Number equations within sections (i.e. 1.1, 1.2, 2.1, 2.2 instead of 1, 2, 3, 4)
%\numberwithin{figure}{section} % Number figures within sections (i.e. 1.1, 1.2, 2.1, 2.2 instead of 1, 2, 3, 4)
%\numberwithin{table}{section} % Number tables within sections (i.e. 1.1, 1.2, 2.1, 2.2 instead of 1, 2, 3, 4)

%\setlength\parindent{0pt} % Removes all indentation from paragraphs - comment this line for an assignment with lots of text

\usepackage{caption}
%\usepackage{topcapt}

\usepackage{booktabs}

\usepackage{graphicx}
\usepackage{adjustbox}
\graphicspath{{../images/}}


%shortcuts for typing variance and expectation
\newcommand{\E}{\mathrm{E}}
\newcommand{\Var}{\mathrm{Var}}

\theoremstyle{definition}
\newtheorem{prob}{Problem}

%----------------------------------------------------------------------------------------
%	TITLE SECTION
%----------------------------------------------------------------------------------------

\newcommand{\horrule}[1]{\rule{\linewidth}{#1}} % Create horizontal rule command with 1 argument of height

\title{
\normalfont \normalsize
\textsc{UPC - Discrete and algorithmic geometry} \\ [25pt] % Your university, school and/or department name(s)
\horrule{0.5pt} \\[0.4cm] % Thin top horizontal rule
\huge Exam project: Enumerate all neighborly $4$-polytopes on $7$ vertices \\ % The assignment title
\horrule{2pt} \\[0.5cm] % Thick bottom horizontal rule
}

\author{Alba Delgado \and Petar Hlad \and Rebeca Guardado \and Simon Van den Eynde} % Your name

\date{\normalsize\today} % Today's date or a custom date

\begin{document}


\maketitle % Print the title


%----------------------------------------------------------------------------------------
%	INTRO
%----------------------------------------------------------------------------------------

\section{Introduction}
%Write here what will do 
Our goal is to solve the following problem:
\begin{prob}
Enumerate all simplicial neighbourly 4-polytopes on 8 vertices up to combinatorial equivalent.
\end{prob}
The rough idea of how we solve this problem is the following. We start from all possible affine Gale diagrams of of a 4-polytope on 8 vertices. These are point configurations of 8 points in the plane, each with an assigned positive or negative sign. We check which of these affine Gale diagrams correspond to neighbourly, simplicial polytopes, and consider only these configurations. Then, check whether they correspond to the same polytope (i.e. to a combinatorially equivalent polytope) and once we have all possible Gale diagrams, retrieve the polytope they correspond to.
\section{Results}
%Write here what the result is
Since this is a studied problem (see \cite{GrSr67}), we know that the result we must obtain is that there are 3 different simplicial neighbourly 4-polytopes on 8 vertices. 
%we can always describe the 3 polytopes here
\section{Discussion}
%Discuss the result here, say why it is special, if it is (un)expected?

\section{Methods}
\subsection{Program structure}
%Describe in short the general programming structure

\subsection{Program details}
%Describe any non-trivial things you programmed, or where you encountered a lot of trouble.

\section{Proof}
%Describe why the program gives the correct result, use theorems and proof them or refer to the proof in a paper/book.

\begin{thebibliography}{9}

\bibitem{GrSr67}
B. Grünbaum, V.P. Sreedharan, An enumeration of simplicial 4-polytopes with 8 vertices. \textit{Journal of Combinatiorial Theory 2}, 437-465, (1967).

\end{thebibliography}
\end{document}